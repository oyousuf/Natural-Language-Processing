\documentclass[wide]{cluu}

\begin{document}
\course{Natural Language Processing}
\author{Author Name}
\title{Assignment 1: Words}
\maketitle

\section{Introduction}

This is an example file of how your submissions should be formatted.
You can either use \LaTeX\ with the \emph{cluu} class to typeset your
submission (recommended) or any other editing device, as long as the
submission is as similar as possible to this example file we gave you.
In this file the option \emph{wide} is used with \emph{cluu} to get
extra wide lines, so the source begins
\begin{verbatim}
\documentclass[wide]{cluu}
\end{verbatim}

\section{\LaTeX}

Typesetting texts in \LaTeX\ is common practice in academic writing.
During the NLP-course you have the chance to develop routine to write
and submit your textual assignments in \LaTeX. You may use the source
file \path{NLP-submission.tex} located in
\path{/local/course/nlp/basic1} and edit it accordingly.

\section{Other}

If you decide not to typeset your submissions in \LaTeX, please make
sure they look similar to this example file. Particularly, please pay
attention to the margins and the font size. After having edited your
submission, please make sure to transform it to \textsc{pdf} format before
submission. We will not accept any other format.

\end{document}
